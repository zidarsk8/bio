% Homework report template for courses lectured by Blaz Zupan.
% For more on LaTeX please consult its documentation pages, or
% read tutorials like http://tobi.oetiker.ch/lshort/lshort.pdf.
%
% Use pdflatex to produce a PDF of a report.

\documentclass[a4paper,11pt]{article}
\usepackage{a4wide}
\usepackage{fullpage}
\usepackage[toc,page]{appendix}
\usepackage[pdftex]{graphicx} % for figures
\usepackage{setspace}
\usepackage{color}
\definecolor{light-gray}{gray}{0.95}
\usepackage{listings} % for inclusion of Python code
\usepackage{hyperref}
\renewcommand{\baselinestretch}{1.2}

\lstset{ % style for Python code, improve if needed
language=Python,
basicstyle=\footnotesize,
basicstyle=\ttfamily\footnotesize\setstretch{1},
backgroundcolor=\color{light-gray},
}

\title{Evolutionary Tree}
\author{Zidar Miha (63060317)}
\date{\today}

\begin{document}

\maketitle

\section{Introduction}

The goal of this assignment was to download and compare a few mitochondrial sequences, based on \textit{COX3} gene, with \textit{Needleman-Wunsch} algorithm and to construct a dendrogram. The algorithm used a \textit{Blosum50} table and a linear gap penalty.

\section{Data}

We used sequences of 14 different animals, show in the Animal table (see Table~\ref{animalTable}), that we downloaded from \url{http://www.ncbi.nlm.nih.gov/genbank/}. From the entire mitochondrial genome, we took out only \textit{COX3} gene for comparison. We also used the \textit{BloSum50} table, for getting the comparison costs for all Amino Acid pairs. You can find all \textit{BloSum} tables at \url{ftp://ftp.ncbi.nih.gov/blast/matrices/}


\begin{table}[htbp]
\caption{Table of animal species used.}
\label{animalTable}
\begin{center}
\begin{tabular}{rllp{6cm}}
\hline
Index & GeneBank id & English name & Latin name\\
\hline
  1 &   NC\_000845.1 &                        pig &                     Sus scrofa \\
  2 &   NC\_004299.1 &              Fugu rubripes &              Takifugu rubripes \\ 
  3 &   AC\_000022.2 &                 Norway rat &              Rattus norvegicus \\
  4 &   NC\_002083.1 &         Sumatran orangutan &                   Pongo abelii \\
  5 &   NC\_001643.1 &                 chimpanzee &                Pan troglodytes \\
  6 &   NC\_011137.1 &                 Neandertal &  Homo sapiens neanderthalensis \\
  7 &   NC\_012920.1 &                      human &                   Homo sapiens \\
  8 &   NC\_001645.1 &            western gorilla &                Gorilla gorilla \\
  9 &   NC\_002008.4 &                        dog &         Canis lupus familiaris \\
 10 &   NC\_006580.1 &                   goldfish &      Carassius auratus auratus \\
 11 &   NC\_012420.1 &           veiled chameleon &          Chamaeleo calyptratus \\
 12 &   NC\_011391.1 &            Russell's viper &               Daboia russellii \\
 13 &   NC\_012061.1 &  Longbeaked common dolphin &             Delphinus capensis \\
 14 &   NC\_001640.1 &                      horse &                 Equus caballus \\
\hline
\end{tabular}
\end{center}
\end{table}


\section{Methods}

We used \textit{Needleman-Wunsch} algorithm with a fixed linear gap penalty $d=-11$ and \textit{Blosum50} scoring matrix, for global alignment of two sequences.\\

Here we have a simple pseudocode implementation of the \textit{Needleman-Wunsch} algorithm

\begin{lstlisting}
  for i=0 to len(s)
    F(i,0) := d*i
  for j=0 to len(t)
    F(0,j) := d*j
  for i=1 to len(s)
    for j=1 to len(t)
    {
      Match := F(i-1,j-1) + S(s[i], t[j])
      Delete := F(i-1, j) + d
      Insert := F(i, j-1) + d
      F(i,j) := max(Match, Insert, Delete)
    }
\end{lstlisting}
Where $s$ and $t$ are input strings, $S$ is the blosum cost table and $F$ is our cost Matrix. The final comparison score is stored in the last element of $F$ matrix on $F[len(s),len(t)]$.\\

Testing the algorithm with different gap penalties didn't show any significant difference, if we kept the penalty withing the absolute bound of the maximum absolute value in the \textit{Blosum50} matrix. 


\section{Results}

Here we have a pairwise score comparions of all 14 animals (see Table~\ref{scoreTab}) and a dendrogram (see Figure-\ref{fig-example}) showing the animals in groups in accordance with their similarity score.


\begin{table}[htbp]
\footnotesize
\label{scoreTab}
\begin{center}
\begin{tabular}{r|rrrrrrrrrrrrrr}
1. & 1814 &      &      &      &      &      &      &      &      &      &      &      &      &     \\
2. & 1574 & 1826 &      &      &      &      &      &      &      &      &      &      &      &     \\
3. & 1641 & 1545 & 1816 &      &      &      &      &      &      &      &      &      &      &     \\
4. & 1592 & 1531 & 1586 & 1804 &      &      &      &      &      &      &      &      &      &     \\
5. & 1613 & 1564 & 1643 & 1703 & 1820 &      &      &      &      &      &      &      &      &     \\
6. & 1597 & 1535 & 1614 & 1702 & 1766 & 1816 &      &      &      &      &      &      &      &     \\
7. & 1622 & 1546 & 1625 & 1713 & 1777 & 1798 & 1823 &      &      &      &      &      &      &     \\
8. & 1603 & 1542 & 1611 & 1727 & 1757 & 1753 & 1764 & 1814 &      &      &      &      &      &     \\
9. & 1665 & 1555 & 1672 & 1567 & 1603 & 1585 & 1608 & 1570 & 1832 &      &      &      &      &     \\
10. & 1582 & 1743 & 1552 & 1551 & 1600 & 1574 & 1585 & 1583 & 1556 & 1824 &      &      &      &     \\
11. & 1305 & 1305 & 1320 & 1264 & 1291 & 1278 & 1299 & 1307 & 1296 & 1305 & 1827 &      &      &     \\
12. & 1401 & 1414 & 1405 & 1386 & 1409 & 1395 & 1406 & 1422 & 1381 & 1417 & 1290 & 1796 &      &     \\
13. & 1698 & 1547 & 1628 & 1580 & 1622 & 1615 & 1640 & 1607 & 1682 & 1553 & 1329 & 1395 & 1826 &     \\
14. & 1718 & 1556 & 1658 & 1585 & 1606 & 1604 & 1615 & 1608 & 1688 & 1564 & 1310 & 1410 & 1685 & 1815\\

\hline
 & 1. & 2. & 3. & 4. & 5. & 6. & 7. & 8. & 9. & 10. & 11. & 12. & 13. & 14. \\

\end{tabular}
\caption{Table comparison scores for all animal pairs. See Animal table \ref{animalTable} for index description}
\end{center}
\end{table}

\pagebreak

\begin{figure}[htbp]
\begin{center}
\includegraphics[scale=0.65]{img/dendrogram-gap-11.png}
\caption{Every figure should include a caption with a figure description.}
\label{fig-example}
\end{center}
\end{figure}


\subsection*{Honor Code}

% The following paragraph of your report should be included as is - do % not change it.

My answers to homework are my own work. I did not make solutions or code available to anyone else. I did not engage in any other activities that will dishonestly improve my results or dishonestly improve/hurt the results of others.

\end{document}
